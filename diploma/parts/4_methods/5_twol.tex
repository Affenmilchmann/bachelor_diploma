\subsection{\texttt{twol} phonology}
Shughni is not rich for morphonological rules. The only global phonological rule is `й' drop (Latin: `y'; IPA: /j/) (\todo{citation needed; `drop' is a questinable name for this}) on some morphemes' borders after a consonant:
\[
j \rightarrow \emptyset 
\ /\ 
[+consonant]
\ \_
\]
\todo{как показать границу морфемы, нужна ли вообще нотация правила если я не могу на неё сослаться}

\todo{Границы морфем принято показывать при помощи `+`}

A special multichar symbol can be introduced to stand in place of `й' letters that are affected by this phonological rule: `\texttt{\{Й\}}'. We capitalize it and frame it with curly brackets so it is never confused with plain letters. A FST treats multichar symbols as single unique symbols, even though they visually consist of multiple characters. This feature allows us to mark which lexicon entries are affected by this phonological rule and which are not. An example of lexicon definition with this symbol is shown in Code block \ref{code:5_1}.

\begin{code_frame}[float]
    \begin{footnotesize}
    \begin{verbatim}
LEXICON PCS # Personal clitics
<1sg>:{Й}ум
<2sg>:{Й}ат
<2sg>:т
<3sg>:{Й}и
<1pl>:{Й}āм
<1pl>:{Й}ам
<2pl>:{Й}ет
<3pl>:{Й}ен

LEXICON DIM # Diminutive clitic
<dim>:{Й}ик
<dim>:{Й}ак
    \end{verbatim}
    \end{footnotesize}
    \tcblower
    \captionof{Code}{A real \texttt{lexd} example of `\texttt{\{Й\}}' multichar usage.}
    \label{code:5_1}
\end{code_frame}

\todo{Выглядит как то, что Вы повторяете себя. Возможно стоит выделить все морфемы типа <2sg>:т в отдельный класс, а дальше гиперфонему {Й} добавить один раз в  разделе Patterns}.

Without \texttt{twol} rules these morphemes will remain as they are specified in \texttt{lexd}, meaning that feeding `\texttt{вирод<n>><dim>}'(\textit{=brother.N-DIM}) to the input of a generator will output literally `\texttt{вирод\{Й\}ик}'. After applying a \texttt{twol} rule (shown in Code block \ref{code:5_2}) it generates wordforms `\texttt{виродик}' (=вирод.N-DIM/brother.N-DIM) and `\texttt{туйик}' (=ту.PERS.2SG-DIM/you.PERS.2SG-DIM). The composition of \texttt{twol} rules with the main FST is the same as it was for morpheme border filtration (Code block \ref{code:bash_8}). This rule applies to all morphemes that can start with `j'. Including, but not limited to personal clitics shown in Code block \ref{code:5_1}, noun adjectivator suffix `-jin' \parencite[168]{parker_shughni_2023}, noun plural suffix `-yen'

\begin{code_frame}[float]
    \begin{footnotesize}
    \begin{verbatim}
"j drop after consonant"
%{Й%}:0 <=> Consonant (%>) (%-) (%>) _ ;
    \end{verbatim}
    \end{footnotesize}
    \tcblower
    \captionof{Code}{A \texttt{twol} rule for `j' drop depending on the previous morpheme's segment. Symbol `\%' in \texttt{twol} is used to escape a character.}
    \label{code:5_2}
\end{code_frame}

\todo{Здесь Вы называете `j' drop --- унифицируйте, пожалуйста.}

There are other phonological rules that do not apply to such variety of contexts:
\begin{itemize}
    \item Two clitics drop a vowel `а' when following a vowel: `=(а)нд' (locative) and `=(а)рд' (dative) \todo{citation needed}. \texttt{twol} rule looks like this:\\
    \texttt{\%\{А\%\}:0 <=> Vowel (\%>) (\%-) (\%>) \_ ;}
    \item A verb PRS suffix `-д/-т' becomes `-д' preceeded by voiced obstruents or vowels and `-т' in all other contexts \parencite[262]{parker_shughni_2023}. \texttt{twol} rule looks like this:\\
    \texttt{\%\{ДТ\%\}:д <=> DGroup \_ ;}\\
    Where \texttt{DGroup} stands for all vowels and voiced obstruents.
\end{itemize}

