\subsection{Repository structure}
\todo{REVIEW THIS SECTION contains duplicate info}\\

For this section all the future references to directories' names the notation \texttt{directory\_name/} stands as an alias for \texttt{repository\_root/directory\_name/}. Meaning if a directory name with no prefix is written its path is assumed relative to the repository root. The repository file structure is shown below. Here only files and directories containing source code are presented.

\vspace*{0.5cm}
\begin{forest}
    for tree={%
        folder,
        grow'=0,
        fit=band,
    }
    [\textbf{Shughni morphology}
        [lexd/]
        [scripts/]
        [translate/]
        [translit/]
        [twol/]
        [Makefile]
    ]
\end{forest}
\vspace*{0.5cm}

\subsubsection*{lexd}
The choice of lexicon compiler was made in favor of \texttt{lexd} as it provides everything that \texttt{lexc} does and in addition has some extra useful functional in form of the tag system, which will be taken advantage of. 

The \texttt{lexd} source code is stored in the \texttt{lexd/} directory. I decided to go with a modular file structure for \texttt{lexd} source code, as it helps to keep the source code organized. The \texttt{lexd/} directory contains \texttt{.lexd} source code files with morpheme lexicons (suffixes, clitics, prefixes, etc.) and lexicon combination patterns. Stem lexicons are stored separately in the \texttt{lexd/lexicons/} directory. For the most part \texttt{lexd/lexicons/} directory contains lexicons obtained from database dumps provided by \textcite{makarov_digital_2022}. The stem lexicon processing is described in Section \ref{stem_lexicons}.

There is no module import feature in \texttt{lexd}. So in order to be able to make a modular \texttt{.lexd} source file structure compilable into a single \texttt{.hfst} file we can concatenate every \texttt{.lexd} module into a single large temporary \texttt{.lexd} file and feed it to the compiler. This is achieved with \texttt{bash} command shown in Code block \ref{code:bash_1}. The \texttt{lexd} compiler outputs FST in AT\&T format and \texttt{hfst-txt2fst} converts it to a binary \texttt{.hfst} file.

\begin{verbbox}
$ cat lexd/*.lexd lexd/lexicons/*.lexd > shughni.lexd
$ lexd shughni.lexd | hfst-txt2fst -o shughni.hfst
\end{verbbox}
{\centering
\vspace*{0.5cm}
\begin{code_frame} \theverbbox \end{code_frame}
\captionof{Code}{Bash command pipeline compiling multiple \texttt{.lexd} files into a single FST.}
\label{code:bash_1}
\vspace*{0.5cm}
}

\subsubsection*{scripts}
The \texttt{scripts/} directory contains various \texttt{Python} scripts and modules that were developed for this project. It includes the source code for metrics evaluation (described in Section \ref{metrics_section}), the source code for converting \texttt{SQL} dumps into \texttt{lexd} lexicons (Section \ref{stem_lexicons} and \ref{rulemm_section}) and the source code for testing FST binary \texttt{.hfst} files (Section \ref{testing_section})

\subsubsection*{translate}
The \texttt{translate/} directory contains \texttt{lexd} source code for Russian lemma translator FST. This is a separate transducer that converts between Cyrillic Shughni stems and Russian lemmas. An example of its work can be seen in Code block \ref{code:bash_2}. Its purpose and source code is described in detail in Section \ref{rulemm_section}.

\begin{verbbox}
$ echo "дарйо<n>" | hfst-lookup -q translate/sgh2rulem.hfst
дарйо<n>    море<n>     0.000000
дарйо<n>    река<n>     0.000000
$ echo "море<n>" | hfst-lookup -q translate/rulem2sgh.hfst 
море<n>     бāр<n>      0.000000
море<n>     дарйо<n>    0.000000
\end{verbbox}
{\centering
\vspace*{0.5cm}
\begin{code_frame} \theverbbox \end{code_frame}
\captionof{Code}{Example of two way FST translator between Shughni stems and Russian lemmas.}
\label{code:bash_2}
\vspace*{0.5cm}
}

\subsubsection*{translit}
The \texttt{translit/} directory contains \texttt{lexd} source code for latin transliterator FST. It is a separate transducer that converts between two scripts: Latin and Cyrillic. An example of its work can be seen in Code block \ref{code:bash_3}. Its purpose and source code is described in detail in Section \ref{translit_section}.

\begin{verbbox}
$ echo "дарйо" | hfst-lookup -q translit/cyr2lat.hfst
дарйо   daryo   0.000000
$ echo "daryo" | hfst-lookup -q translit/lat2cyr.hfst
daryo   дарйо   0.000000
\end{verbbox}
{\centering
\vspace*{0.5cm}
\begin{code_frame} \theverbbox \end{code_frame}
\captionof{Code}{Example of two way FST transliterator between Cyrillic and Latin scripts.}
\label{code:bash_3}
\vspace*{0.5cm}
}

\subsubsection*{twol}
The \texttt{twol/} directory contains \texttt{.twol} source code files. Shughni has very few morphonological rules \todo{describe it here?}