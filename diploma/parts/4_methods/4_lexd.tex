\subsection{\texttt{lexd} rule declaration}
The choice of lexicon compiler was made in favor of \texttt{lexd} as it provides everything that \texttt{lexc} does and in addition has some extra useful functional in form of the tag system, which will be taken advantage of. 

The \texttt{lexd} source code is stored in the \texttt{lexd/} directory. I decided to go with a modular file structure for \texttt{lexd} source code, as it helps to keep the source code organized. The \texttt{lexd/} directory contains \texttt{.lexd} source code files with morpheme lexicons (suffixes, clitics, prefixes, etc.) and lexicon combination patterns. Stem lexicons are stored separately in the \texttt{lexd/lexicons/} directory. For the most part \texttt{lexd/lexicons/} directory contains lexicons obtained from database dumps provided by \textcite{makarov_digital_2022}. The stem lexicon processing is described in Section \ref{stem_lexicons}.

There is no module import feature in \texttt{lexd}. So in order to be able to make a modular \texttt{.lexd} source file structure compilable into a single \texttt{.hfst} file we can concatenate every \texttt{.lexd} module into a single large temporary \texttt{.lexd} file and feed it to the compiler. This is achieved with \texttt{bash} command shown in Code block \ref{code:4_1}. The \texttt{lexd} compiler outputs FST in AT\&T format and \texttt{hfst-txt2fst} converts it to a binary \texttt{.hfst} file.

\begin{code_frame}[float]
    \begin{footnotesize}
    \begin{verbatim}
$ cat lexd/*.lexd lexd/lexicons/*.lexd > sgh.lexd
$ lexd sgh.lexd | hfst-txt2fst -o sgh_base_stem.hfst
    \end{verbatim}
    \end{footnotesize}
    \tcblower
    \captionof{Code}{Bash command pipeline compiling multiple \texttt{.lexd} files into a single FST.}
    \label{code:4_1}
\end{code_frame}

\todo{Finish when done!!}
\subsubsection*{Nouns}
\subsubsection*{Verbs}
\subsubsection*{Adjectives}
\subsubsection*{Pronouns}
\subsubsection*{Numerals}
\subsubsection*{Anything else(?)}