\subsection{Input and output format}
Every FST converts between two types of strings: wordforms (e.g. `\texttt{дарйойен}' $\approx$ `rivers') and glossed strings (e.g. `\texttt{дарйо<n>><pl>}' = `\texttt{дарйо.N-PL}'). Glossed strings format is based on Apertium format \todo{citation needed?}, which is a standardized format for HFST-based transducers. The choice of this format is motivated by the fact that this is the standard, and I want to keep it consistent with existing tools and practices in the field. Important keys of the format are:

\todo{Нужно собрать список глосс используемых в тексте и отправить списком в начало. Я ожидаю что-то такое в алфавитном порядке. Список и значение глосс достаточно стандартизирован, и можете взять из моего пакета `lingglosses` (прислал в личку). Еще лингвисты обычно пишут глоссы смолл кэпсом \textsc{}, но можно, наверное, на это забить. 

- PL <pl> --- plural marker.}

\begin{itemize}
    \item Grammatical tags are enclosed with angular brackets and are lowercase.\\
    \texttt{stone.V} = \texttt{stone<v>}
    \item Part of speech tags (POS) are obligatory and stand next to the stem or lemma.\\
    \texttt{stone.SG} = \texttt{stone<n><sg>}
    \item Different morpheme tags are separated with a single right angular bracket `\texttt{>}'.\\
    \texttt{stone-PL} = \texttt{stone<n>><pl>}
    \item Multiple stems in a single word are possible.\\
    \texttt{lemma1<adj>><morph>>lemma2<adj>}
\end{itemize}

The Shughni morphological parser in this work is presented in a variety of input and output formats. A full list of \texttt{.hfst} files with their formats of input and output is shown in Table \ref{Tab:all_hfst}. Motivation for this many variations of the morphological parser was support of different formats. Parser comes with four format variables:
\begin{itemize}
    \item FST directionality: analyzer or generator. This simply shows the direction of a FST, analyzers take wordforms as input and return glossed strings, generators take glossed strings and return wordforms
    \item Glossed string stem glosses: Shughni stems or Russian lemmas. Notated in file names as \texttt{stem} and \texttt{rulem} respectively.\\
    \texttt{дарйо<n>><pl>} (Shughni stem); \texttt{река<n>><pl>} (Russian lemma)
    \item Wordform morpheme borders: plain word or morphemes are separated with (`\texttt{>}') delimeter symbol. Notated in file names as \texttt{word} and \texttt{morph} respectively.\\ \todo{я за то, чтобы называть дарйо>йен segmented, а инструмент, который это делает segmenter, так что может быть тэг morph я бы тоже заменил.}
    \texttt{дарйойен} (plain word); \texttt{дарйо>йен} (morphemes separated)
    \item Wordform script: Latin or Cyrillic.\\
    \texttt{дарйойен} (Latin); \texttt{daryoyen} (Cyrillic)
\end{itemize}

\begin{table}[!h]
    \begin{center}
        \begin{tabular}{|l|l|l|}
            \hline
            \textbf{Transducer file name} & \textbf{Input example} & \textbf{Output example} \\
            \hline
            sgh\_gen\_stem\_morph\_cyr.hfst & \texttt{дарйо<n>><pl>} & \texttt{дарйо>йен} \\
            sgh\_gen\_stem\_word\_cyr.hfst & \texttt{дарйо<n>><pl>} & \texttt{дарйойен} \\
            sgh\_gen\_rulem\_morph\_cyr.hfst & \texttt{река<n>><pl>} & \texttt{дарйо>йен} \\
            sgh\_gen\_rulem\_word\_cyr.hfst & \texttt{река<n>><pl>} & \texttt{дарйойен} \\
            sgh\_gen\_stem\_morph\_lat.hfst & \texttt{дарйо<n>><pl>} & \texttt{daryo>yen} \\
            sgh\_gen\_stem\_word\_lat.hfst & \texttt{дарйо<n>><pl>} & \texttt{daryoyen} \\
            sgh\_gen\_rulem\_morph\_lat.hfst & \texttt{река<n>><pl>} & \texttt{daryo>yen} \\
            sgh\_gen\_rulem\_word\_lat.hfst & \texttt{река<n>><pl>} & \texttt{daryoyen} \\
            sgh\_analyze\_stem\_morph\_cyr.hfst & \texttt{дарйо>йен} & \texttt{дарйо<n>><pl>} \\
            sgh\_analyze\_stem\_word\_cyr.hfst & \texttt{дарйойен} & \texttt{дарйо<n>><pl>} \\
            sgh\_analyze\_rulem\_morph\_cyr.hfst & \texttt{дарйо>йен} & \texttt{река<n>><pl>} \\
            sgh\_analyze\_rulem\_word\_cyr.hfst & \texttt{дарйойен} & \texttt{река<n>><pl>} \\
            sgh\_analyze\_stem\_morph\_lat.hfst & \texttt{daryo>yen} & \texttt{дарйо<n>><pl>} \\
            sgh\_analyze\_stem\_word\_lat.hfst & \texttt{daryoyen} & \texttt{дарйо<n>><pl>} \\
            sgh\_analyze\_rulem\_morph\_lat.hfst & \texttt{daryo>yen} & \texttt{река<n>><pl>} \\
            sgh\_analyze\_rulem\_word\_lat.hfst & \texttt{daryoyen} & \texttt{река<n>><pl>} \\
            \hline
        \end{tabular}
        \caption{A full list of available HFST transducers}
        \label{Tab:all_hfst}
    \end{center}
\end{table}

Four binary variables result in 16 ($=2^4$) FST variations. Every FST listed in Table \ref{Tab:all_hfst} can be built with \texttt{make} command as shown in Code block \ref{code:make_1}. Regular \texttt{.hfst} binary FSTs are not recommended using in production environments, as they are not optimized. For production use an optimized format called \texttt{.hfstol} (HFST \textbf{O}ptimized \textbf{L}ookup), which can also be compiled automatically for every listed FST using \texttt{make}.

\begin{code_frame}[float]
    \begin{footnotesize}
    \begin{verbatim}
$ make sgh_gen_stem_morph_cyr.hfst
$ make sgh_gen_stem_morph_cyr.hfstol
    \end{verbatim}
    \end{footnotesize}
    \tcblower
    \captionof{Code}{Example of FST compilation with \texttt{make}.}
    \label{code:make_1}
\end{code_frame}
