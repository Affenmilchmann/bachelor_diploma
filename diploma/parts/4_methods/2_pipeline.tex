\subsection{FST compilation pipeline}
\subsubsection*{Analyzers and generators}
First, there are two main types of FSTs: generators and analyzers. They differ only in the directionality of a FST. Analyzers take wordforms as input and return glossed strings as output. Generators work in reverse, as shown in Code block \ref{code:pipeline_1}.

\begin{code_frame}[float]
    \begin{footnotesize}\codespacing
    \begin{verbatim}
$ echo "дарйойен" | hfst-lookup -q sgh_analyze_stem_word_cyr.hfst
дарйойен        дарйо<n>><3pl>  0.000000
дарйойен        дарйо<n>><pl>   0.000000
$ echo "дарйо<n>><pl>" | hfst-lookup -q sgh_gen_stem_word_cyr.hfst
дарйо<n>><pl>   дарйо-йен       0.000000
дарйо<n>><pl>   дарйойен        0.000000
    \end{verbatim}
    \end{footnotesize}
    \tcblower
    \captionof{Code}{FST analyzer vs generator output formats.}
    \label{code:pipeline_1}
\end{code_frame}

The \texttt{lexd} source code is written as a generator, meaning by default, compiled FST takes glossed stem or lemma as input and returns a wordform. To compile any analyzer, a corresponding generator is inverted, as shown in Code block \ref{code:pipeline_2}.

\begin{code_frame}[float]
    \begin{footnotesize}\codespacing
    \begin{verbatim}
$ hfst-invert sgh_gen_stem_word_cyr.hfst  \
           -o sgh_analyze_stem_word_cyr.hfst
    \end{verbatim}
    \end{footnotesize}
    \tcblower
    \captionof{Code}{FST analyzer creation from a FST generator.}
    \label{code:pipeline_2}
\end{code_frame}

\subsubsection*{Shughni stems and Russian lemmas}
The next format only applies to the glossed side of FSTs (to the analyzers' output and to the generators' input). It sets whether a Cyrillic Shughni stem or a Russian translated lemma will be used as a stem's gloss, as shown in Code block \ref{code:pipeline_3}. Shughni stems can have multiple Russian candidates (\textit{`дарйо'} can be translated as \textit{`река'} or \textit{`море'}). This leads to composed transducer having more output candidates. This works both ways, meaning Russian lemmas can translate as multiple Shughni stems (\textit{`река'} can be translated as \textit{`дарйо'} or \textit{`х̌ац'}).

\begin{code_frame}[float]
    \begin{footnotesize}\codespacing
    \begin{verbatim}
$ echo "дарйо<n>><3pl>" | hfst-lookup -q sgh_gen_stem_word_cyr.hfst
дарйо<n>><3pl>      дарйо-йен       0.000000
дарйо<n>><3pl>      дарйойен        0.000000
$ echo "река<n>><3pl>" | hfst-lookup -q sgh_gen_rulem_word_cyr.hfst
река<n>><3pl>       дарйо-йен       0.000000
река<n>><3pl>       дарйойен        0.000000
река<n>><3pl>       х̌ац-ен          0.000000
река<n>><3pl>       х̌ацен           0.000000
$ echo "дарйойен" | hfst-lookup -q sgh_analyze_rulem_word_cyr.hfst
дарйойен            море<n>><3pl>   0.000000
дарйойен            море<n>><pl>    0.000000
дарйойен            река<n>><3pl>   0.000000
дарйойен            река<n>><pl>    0.000000
    \end{verbatim}
    \end{footnotesize}
    \tcblower
    \captionof{Code}{Shughni stem vs Russian lemma versions of FST.}
    \label{code:pipeline_3}
\end{code_frame}

The \texttt{lexd} source code contains lexicons with Shughni stems on the glossed side, meaning default compiled FST contains only Shughni stems. The process of creating a FST that works with Russian lemmas on the glossed side is more complicated. It is achieved with the help of a second FST \texttt{rulem2sgh.hfst}, its only purpose is translating stems to lemmas. It is attached to the input of a generator FST, creating a pipeline 

\begin{center}
\noindent `\texttt{река<n>><pl>}' $\rightarrow$ \texttt{rulem2sgh} $\rightarrow$ `\texttt{дарйо<n>><pl>}' $\rightarrow$ \texttt{generator} $\rightarrow$ `\texttt{дарйойен}'\\
\end{center}


It can be done with `compose' transducer operation (see Code block \ref{code:bash_7}), which takes two FSTs, directs first's output to the second's input and returns the resulting composed FST. Details of the translator FST's development are described in Section \ref{rulemm_section}.

\begin{code_frame}[float]
    \begin{footnotesize}\codespacing
    \begin{verbatim}
$ hfst-compose rulem2sgh.hfst sgh_gen_stem_word_cyr.hfst
  -o sgh_gen_rulem_word_cyr.hfst
    \end{verbatim}
    \end{footnotesize}
    \tcblower
    \captionof{Code}{Shughni stem translator composition.}
    \label{code:bash_7}
\end{code_frame}

\subsubsection*{Latin script support}
The source code contains only Cyrillic lexicons. The support of Latin script comes with the help of a separate transliterator FST, as it was the case for Russian lemmas support. Transliteration is only applied to the wordform side of FSTs. The glossed side's stems and lemmas are left Cyrillic. The pipeline for transliteration is shown below:

\begin{center}
\noindent `\texttt{дарйо<n>><pl>}' $\rightarrow$ \texttt{generator} $\rightarrow$ `\texttt{дарйойен}' $\rightarrow$ \texttt{cyr2lat} $\rightarrow$ `\texttt{daryoyen}'\\
\noindent `\texttt{daryoyen}' $\rightarrow$ \texttt{lat2cyr} $\rightarrow$ `\texttt{дарйойен}' $\rightarrow$ \texttt{analyzer} $\rightarrow$ `\texttt{дарйо<n>><pl>}'\\
\end{center}

The compilation process is the same as it is for Russian lemmas shown in Code block \ref{code:bash_7}. The only difference is that transliterator is applied to the wordform side of a FST, while Russian lemma translator is applied to the glossed side. See Section \ref{translit_section} for more details about the transliterator development.

\subsubsection*{Wordform segmentation} \label{morpheme_borders}
Theoretical linguists sometimes need wordforms to have morphemes separated by a special symbol that does not appear naturally in a language. In this morphological parser I choose the right angular bracket symbol '\texttt{>}' for this role to keep it consistent with glossed strings morpheme separator. This presented a slight challenge to the development, as regular wordforms often may contain hyphens (`-') between some morphemes. For an example, `\texttt{дарйойен}' can also be spelled as `\texttt{дарйо-йен}'. But for the morpheme separated FST wordform must contain only `\texttt{дарйо>йен}', with no extra optional hyphens. 

This was solved with the help of \texttt{twol} rules. The base \texttt{lexd} FST (\texttt{sgh\_base\_stem.hfst}) contains both hyphens and morpheme separators, which looks like `\texttt{дарйо>-йен}' and `\texttt{дарйо>йен}' on the wordform side. Then the filtering is done with help of two FSTs that contain \texttt{twol} rules shown in Code block \ref{code:twol_4_1}. The \texttt{sep.hfst} removes all separator symbols (`\texttt{>}') from the wordform, leaving `\texttt{дарйо-йен}' and `\texttt{дарйойен}'. And \texttt{hyphen.hfst} removes all hyphens from the wordform, leaving `\texttt{дарйо>йен}' and `\texttt{дарйо>йен}', which then fold into a single FST path after optimization.

\begin{code_frame}[float,floatplacement=!h]
    \begin{footnotesize}\codespacing
    \begin{verbatim}
# hyphen.twol
Rules
"Hyphen removal to avoid '>-'/'->' situations"
%-:0 <=> _ ;

# sep.twol
Rules
"Morpeme separator removal"
%>:0 <=> _ ;
    \end{verbatim}
    \end{footnotesize}
    \tcblower
    \captionof{Code}{\texttt{twol} rules that filter morpheme border.}
    \label{code:twol_4_1}
\end{code_frame}

For this case compose-intersect operation is applied, that allows \texttt{twol}-compiled FSTs to be composed with regular lexicon \texttt{lexd} FSTs. Compilation command is shown in Code block \ref{code:bash_8}.

\begin{code_frame}[float,floatplacement=!h]
    \begin{footnotesize}\codespacing
    \begin{verbatim}
$ hfst-compose-intersect sgh_base_stem.hfst twol/sep.hfst 
  -o sgh_gen_stem_word_cyr.hfst
$ hfst-compose-intersect sgh_base_stem.hfst twol/hyphen.hfst 
  -o sgh_gen_stem_morph_cyr.hfst
    \end{verbatim}
    \end{footnotesize}
    \tcblower
    \captionof{Code}{Plain text and morpheme separated text FST versions compilation.}
    \label{code:bash_8}
\end{code_frame}

This way, every FST listed in Table \ref{Tab:all_hfst} can be compiled using a specific combination of bash HFST tools shown in this section.

%\begin{forest}
%    for tree={%
%      folder,
%      grow'=0,
%      fit=band,
%    }
%    [sgh\_base\_stem.hfst
%        [sgh\_gen\_stem\_morph\_cyr.hfst
%          [sgh\_gen\_stem\_morph\_lat.hfst
%            [sgh\_analyze\_stem\_morph\_lat.hfst]
%          ]
%          [sgh\_analyze\_stem\_morph\_cyr.hfst]
%        ]
%        [sgh\_gen\_stem\_word\_cyr.hfst
%          [sgh\_gen\_stem\_word\_lat.hfst
%            [sgh\_analyze\_stem\_word\_lat.hfst]
%          ]
%          [sgh\_analyze\_stem\_word\_cyr.hfst]
%        ]
%        [sgh\_base\_rulem.hfst
%          [sgh\_gen\_rulem\_morph\_cyr.hfst
%            [sgh\_gen\_rulem\_morph\_lat.hfst
%              [sgh\_analyze\_rulem\_morph\_lat.hfst]
%            ]
%            [sgh\_analyze\_rulem\_morph\_cyr.hfst]
%          ]
%          [sgh\_gen\_rulem\_word\_cyr.hfst
%            [sgh\_gen\_rulem\_word\_lat.hfst
%              [sgh\_analyze\_rulem\_word\_lat.hfst]
%            ]
%            [sgh\_analyze\_rulem\_word\_cyr.hfst]
%          ]
%        ]
%    ]
%\end{forest}
%\captionof{Code}{Plain text and morpheme separated text FST versions compilation.}