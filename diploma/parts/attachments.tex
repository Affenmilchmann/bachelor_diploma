\newpage
\section*{Appendix}
\addcontentsline{toc}{section}{Appendix}
\subsection*{Links}
\addcontentsline{toc}{subsection}{Links}
\label{github_repo}
The complete source code, including all HFST building dependencies were made available under an open-source license at:
\url{https://github.com/afmigoo/shughni_morphology}

\newpage
\subsection*{Tables}
\addcontentsline{toc}{subsection}{Tables}
\begin{table}[!h]
    \begin{center}
        \caption{A full list of glosses used in this work. Part 1.}
        \label{Tab:glosses_1}
        \begin{tabular}{|p{4cm}|p{3cm}|p{8.5cm}|}
            \hline
            \textbf{FST tag} & \textbf{Gloss} & \textbf{Meaning} \\
            \hline
            \hline
            \texttt{<1pl>} & 1PL & First-person plural\\
            \texttt{<1sg>} & 1SG & First-person singular\\
            \texttt{<2pl>} & 2PL & Second-person plural\\
            \texttt{<2sg>} & 2SG & Second-person singular\\
            \texttt{<3pl>} & 3PL & Third-person plural\\
            \texttt{<3sg>} & 3SG & Third-person singular\\
            \texttt{<add1>}, \texttt{<add2>} & ADD & Additive (e.g. `-ga')\\
            \texttt{<adv>} & ADV & Adverbializer suffix\\
            \texttt{<advs>} & ADVS & Adversative conjunction\\
            \texttt{<agn>} & AGN & Verb agent nominalization\\
            \texttt{<aug>} & AUG & Adjective intensifier\\
            \texttt{<cause>} & CAUSE & ($\approx$ `purpose'/`benefit'/`cause')\\
            \texttt{<com>} & COM & Comitative ('together with', 'in the company of') or instrumental case\\
            \texttt{<comp>} & COMP & Adjective comparative\\
            \texttt{<comp\_at>} & COMP\_AT & An attenuative comparative of an adjective\\
            \texttt{<compl>} & COMPL & Сompletive conjunction\\
            \texttt{<cont1>}, \texttt{<cont2>} & CONT & Continuative (usually used with body part nouns)\\
            \texttt{<coord1>} & COORD1 & Coordinating conjunction `=xu'\\
            \texttt{<coord2>} & COORD2 & Coordinating conjunction `='/`=ad'\\
            \texttt{<coord3>} & COORD3 & Tajik coordinating conjunction `=u'\\
            \texttt{<d1>} & D1 & Proximal demonstrative\\
            \texttt{<d2>} & D2 & Medial demonstrative\\
            \texttt{<d3>} & D3 & Distal demonstrative\\
            \texttt{<dat>} & DAT & Dative case\\
            \texttt{<dim>} & DIM & Diminutive noun suffix\\
            \texttt{<dir>} & DIR & Directive `in direction of'\\
            \texttt{<disj>} & DISJ & Disjunctive conjunction\\
            \texttt{<emph>} & EMPH & Emphasizing\\
            \texttt{<f>} & F & Feminine gender\\
            \texttt{<f\_pl>} & F/PL & Feminine gender or plural number\\
            \texttt{<full>} & FULL & Adjective suffix `=full of X'\\
            \texttt{<fut>} & FUT & Future tense, habitual aspect\\
            \texttt{<hb>} & HB & Habilitive noun suffix\\
            \texttt{<hon>} & HON & Honorable personal pronoun suffix\\
            \texttt{<imp>} & IMP & Verb imperative stem\\
            \texttt{<in>} & IN & Noun suffix `in'/`at' (e.g. `in a house')\\
            \texttt{<inf>} & INF & Verb infinitive stem\\
            \texttt{<int>} & INT & Intensifier\\
            \texttt{<lim1>}, \texttt{<lim2>} & LIM2 & Spacial and temporal limitative, instrumental case\\
            \hline
        \end{tabular}
    \end{center}
\end{table}

\begin{table}[!h]
    \begin{center}
        \caption{A full list of glosses used in this work. Part 2.}
        \label{Tab:glosses_2}
        \begin{tabular}{|p{4cm}|p{3cm}|p{8.5cm}|}
            \hline
            \textbf{FST tag} & \textbf{Gloss} & \textbf{Meaning} \\
            \hline
            \hline
            \texttt{<loc>} & LOC & Locative\\
            \texttt{<m>} & M & Masculine gender\\
            \texttt{<neg.q>} & NEG.Q & Negative question particle\\
            \texttt{<neg>} & NEG & Negative prefix\\
            \texttt{<obl>} & OBL & Oblique case\\
            \texttt{<ord>} & ORD & Ordinal numeral suffix\\
            \texttt{<orig>} & ORIG & `Origin' noun suffix\\
            \texttt{<p.loc>} & P.LOC & Possessive locative\\
            \texttt{<pl>} & PL & Plural number\\
            \texttt{<place>} & PLACE & `Place' noun suffix\\
            \texttt{<pqp>} & PQP & Plusquamperfect verb suffix\\
            \texttt{<prf>} & PRF & Perfect tense\\
            \texttt{<proh>} & PROH & Prohibitive verb prefix\\
            \texttt{<prol>} & PROL & Prolative ($\approx$ `along'/`through')\\
            \texttt{<prs>} & PRS & Non-past tense\\
            \texttt{<pst>} & PST & Past tense\\
            \texttt{<ptcp1>} & PTCP1 & Adjective resultative participle suffix\\
            \texttt{<ptcp2>} & PTCP2 & Verbal resultative participle suffix\\
            \texttt{<purp>} & PURP & Purposive suffix of verb infinitive stem\\
            \texttt{<q>} & Q & Question particle\\
            \texttt{<redup>} & REDUP & Reduplication\\
            \texttt{<refl>} & REFL & Reflexive pronoun\\
            \texttt{<self>} & SELF & Self pronoun\\
            \texttt{<sg>} & SG & Singular number\\
            \texttt{<subd>} & SUBD & Subordinating conjunction\\
            \texttt{<subst>} & SUBST & Substantivizing suffix\\
            \texttt{<sup>} & SUP & Superessive case\\
            \texttt{<time>} & TIME & Noun suffix denoting time (e.g. `at night')\\
            \texttt{<with>} & WITH & Adjective suffix `=with X'\\
            \texttt{<with\_little>} & WITH\_LITTLE & Adjective suffix `=with little X'\\
            \texttt{<without>} & WITHOUT & Adjective suffix `=without X'\\
            \hline
        \end{tabular}
    \end{center}
\end{table}

\begin{table}[!h]
    \begin{center}
        \caption{A full list of POS (part of speech) tags used in this work and FST.}
        \label{Tab:pos_tags}
        \begin{tabular}{|c|l|l|}
            \hline
            \textbf{FST tag} & \textbf{Gloss} & \textbf{Meaning} \\
            \hline
            \hline
            \texttt{<adj>} & ADJ & Adjective\\
            \texttt{<conj>} & CONJ & Conjunction\\
            \texttt{<dem>} & D (\textit{is ignored in glosses, as D1, D2 or D3 is always present}) & Demonstrative\\
            \texttt{<intj>} & INTJ & Interjection\\
            \texttt{<n>} & N & Noun\\
            \texttt{<num>} & NUM & Numeral\\
            \texttt{<pers>} & pers & Personal pronoun\\
            \texttt{<post>} & post & Postposition\\
            \texttt{<prep>} & prep & Preposition\\
            \texttt{<pro>} & pro & Pronoun\\
            \texttt{<prt>} & prt & Particle\\
            \texttt{<v>} & v & Verb\\
            \hline
        \end{tabular}
    \end{center}
\end{table}

\newpage