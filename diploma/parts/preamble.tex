\documentclass[12pt]{article}
\usepackage[left=2.5cm, right=2.5cm, top=1.5cm, bottom=2cm]{geometry} % поля страницы
\usepackage[russian,english]{babel}

% шрифт
\babelfont{rm}{FreeSerif}
\babelfont{sf}{FreeSans}
\babelfont{tt}[Scale=1.0]{FreeMono}
\usepackage{fontspec}
\defaultfontfeatures{Ligatures={TeX},Renderer=Basic}

% изображения
\usepackage[dvips,xetex]{graphicx}
% \usepackage{ifpdf,mla}% <-- mla.sty requires ifpdf.sty, but (perversely) doesn't load it

% лингвистические примеры
\usepackage{expex}
\lingset{belowglpreambleskip=0ex, aboveglftskip=0ex, everygla={\upshape}}
\usepackage[edges]{forest} % trees

\usepackage{fancyhdr} % установка колонтитулов
\usepackage{setspace} % для отступов
\usepackage[section]{placeins} % чтоб картинки из секций не улетали
\usepackage{indentfirst} % apply indents to fist paragraphs
\usepackage{hyperref} % clickable table of contents
\usepackage{array, makecell} % force split cell text
\usepackage{diagbox} % diagonal splitted table cell

% Библиограифя
\usepackage{csquotes}
\usepackage[backend=biber, style=apa]{biblatex}
\addbibresource{\rootdir/ref.bib}

% Настройка стиля страницы
\pagestyle{fancy}      % Использование стиля "fancy" для оформления страниц
\fancyhf{}              % Очистка текущих значений колонтитулов
\fancyfoot[C]{\thepage} % Установка номера страницы в нижнем колонтитуле по центру
\newcommand{\globalspacing}{\linespread{1.5}}
\newcommand{\codespacing}{\linespread{0.9}}
\newcommand{\bibspacing}{\linespread{1.0}}

% Для оформления временных сносок
\usepackage{xcolor}    % для работы с цветами
\newcommand{\todo}[1]{\textbf{\textcolor{red}{TODO: #1}}}

%%% БЛОКИ КОДА %%%
\usepackage[most]{tcolorbox}
\usepackage{verbatimbox,caption,float,lipsum} % https://tex.stackexchange.com/questions/581714/how-to-use-caption-in-a-verbatim-environment
% Счётик для блоков кода
\newfloat{Code}
\captionsetup{Code}
\captionsetup[Code]{belowskip=0pt, aboveskip=0pt}
\preto{\verbatim}{\topsep=0pt \partopsep=0pt}
% Кастомные коробки для блоков кода
\newtcolorbox{code_frame}[1][]{ % https://tex.stackexchange.com/questions/263706/latex-adding-cell-padding-to-a-text-frame
  enhanced,
  arc=4pt,
  outer arc=4pt,
  colback=white,
  boxrule=0.9pt,
  #1
}
\newtcolorbox{subbox}[1][]{
  enhanced,
  arc=0pt,
  outer arc=0pt,
  colframe=white,
  coltitle=black,
  boxrule=0.9pt,
  left=0mm,top=1mm,bottom=1mm,right=0mm,boxsep=1mm,width=7.5cm,nobeforeafter,
  height=5cm,
  #1}
% Template
% \begin{code_frame}[float]
%     \begin{footnotesize}\codespacing
%     \begin{verbatim}
% C O D E
%     \end{verbatim}
%     \end{footnotesize}
%     \tcblower
%     \captionof{Code}{Title}
%     \label{code:label}
% \end{code_frame}