\section{Introduction}

\subsection{Shughni}
The Shughni language (ISO: sgh; Glottolog: shug1248) is a language of the Iranian branch of the Indo-European family \parencite[12]{plungian_study_2022}. As of June 1997, it was estimated to be spoken by approximately 100,000 people \parencite[225]{edelman_languages_1999} \todo{найти http://old.iea.ras.ru/publications\_new/kalandarov.html ? там тоже есть оценка} in the territories of Tajikistan and Afghanistan. Both countries have a subregion where Shughni is the most widely spoken native language. The Shughni-speaking subregion of Tajikistan is called `Shughnon' and it belongs the to `Gorno-Badakhshan Autonomus' province. In Afghanistan, the Shughni-speaking region is called `Shughnan' and it lies within the territory of `Badakhshan' province \parencite[2]{parker_shughni_2023}.

\todo{Pamiri}
\subsection{Morphology parsing \todo{rethink this heading}}