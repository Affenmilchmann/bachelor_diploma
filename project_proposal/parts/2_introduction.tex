\section{Introduction}

\par Morphological parser is a fundamental tool, a wide range of computational linguistics' tasks rely on some form of morphological model. For morphologically rich languages it is close to impossible to list and manually define all the possible word-forms. The only reasonable way to approach such problem is to model a language's morphology. 
\par Shughni language (ISO: sgh; glottolog: shug1248) is a morphologically rich low-resource language. It belongs to the Iranian branch of the Indo-European family, and it is spoken by circa 100 000 people (Edelman and Dodykhudoeva, 2009) in two regions: Mountainous Badakhshan Autonomous Region (Tajikistan) and Badakhshan Province (Afghanistan). Shughni has a mixed morphological typology type (Parker 2023: 94), which means that grammatical meanings can be carried by morphemes, words or clitics. There are three scripts in Shughni language: Latin, Cyrillic and Arabic. The Arabic script is used on the territory of Badakhshan Province of Afghanistan, and Cyrillic and Latin scripts are used in the Mountainous Badakhshan Autonomous Region of Afghanistan.
\par \todo{Указать какую версию шугнанского я покрываю, почему, и сослаться на https://aclanthology.org/2022.eurali-1.9.pdf}
\par For high-resource languages this problem is usually being solved using deep learning (DL) models, which require large amounts of training data. This method is not available for low-resource languages that lack digital text data. For such languages linguists usually apply rule-based approach.