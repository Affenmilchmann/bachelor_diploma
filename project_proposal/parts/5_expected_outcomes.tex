\section{Expected outcomes}
\par The main product of this work is a set of finite-state transducers capable of morphological analysis and morphological generation. This tool set then can be used for different NLP tasks such as POS-tagging, spell-checking, morphological tagging or even translation. In the Table \hyperref[Tab:format]{1} the planned input-output formats of the FSTs are presented.
\begin{table}[!h]
    \begin{center}
        \begin{tabular}{|l|l|l|} 
            \hline
            Description & Input example & Output example \\
            \hline
            \hline
            Cyrillic plain analyser & дарйойен & дарйо<n>{}><pl> \\
            \hline
            Cyrillic morph analyser & дарйо>йен & дарйо<n>{}><pl> \\
            \hline
            Latin plain analyser & daryoyen & дарйо<n>{}><pl> \\
            \hline
            Latin morph analyser & daryo>yen & дарйо<n>{}><pl> \\
            \hline
            Cyrillic plain generator & дарйо<n>{}><pl> & дарйойен \\
            \hline
            Cyrillic morph generator & дарйо<n>{}><pl> & дарйо>йен \\
            \hline
            Latin plain generator & дарйо<n>{}><pl> & daryoyen \\
            \hline
            Latin morph generator & дарйо<n>{}><pl> & daryo>yen \\
            \hline
        \end{tabular}
        \label{Tab:format}
        \caption{A full list of planned HFST transducers}
    \end{center}
\end{table}
`Plain' analysers and generators work with the word-forms as they are occurring in plain text, while `morph' analysers and generators work with word-forms that have morphemes separated by `>' symbol. The latter word-form format is useful to linguists in research, while the first format is easily applicable to plain texts.
\par The title of this work mentions only verbs, nouns and adjectives, however this morphological parser is planned to cover all parts of speech. The title is focused on the verbs, nouns and adjectives since their morphological paradigms are the biggest. In terms of parser's quality I aim to reach 80\% \textit{coverage}, the best case scenario would be more than 90\%. The existing non HFST-based parser for Shughni created by \textcite{melchenko_2021_parser} has final recall score of 90\%, which formula is equivalent to my \textit{coverage} and Melchenko's \textit{accuracy} score is 69\%. For HFST-based parsers for other languages that were mentioned earlier \textit{coverage} varies from 40\% \parencite{buntyakova_2023_twol} to 90\% \parencite{ivanova_free_2022}. My target \textit{coverage} values are motivated by the results of existing solution for Shughni and by the results of my previous work \parencite{osorgin_2024_twol} where I have reached the \textit{coverage} value of 45\% by implementing nouns, pronouns, numerals and prepositions.
\par The source code and the compiled \texttt{.hfst} files will be uploaded to the public GitHub repository for anyone to use and contribute to.